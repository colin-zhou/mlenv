% !TEX encoding = UTF-8
% !TEX program = lualatex

\documentclass{article}
\usepackage{color}
\usepackage{indentfirst}
\usepackage{luatexja-fontspec}
\setmainjfont{FandolSong}


\title{C Coding Tips}
\author{xuxx}
\date{2016-08-25}

\begin{document}
\begin{titlepage}
\maketitle
\end{titlepage}

\section{编码规范}
\begin{enumerate}
        \item 推荐参考Linux Kernel编码风格/或BSD C/C++编码风格,没有强制的要求.
                参考: Kernel源码目录: Documentation/CodingStyle
        \item 建议:
            \begin{enumerate}
            \item Tab缩进为8字符;
            \item if语句使用\{\}包围,尤其对于单行语句;
            \item 单行最大字符为80字符,C++代码不超过100字符;
            \end{enumerate}
        \item 代码格式化工具:
            \begin{enumerate}
            \item Visual Studio
            \item vim '='
            \item asytle [astyle.sourceforge.net]
                    参数:
                    astyle --style=kr -s8 -S -N -w -Y -U -k3 -W3 -J -xC80 -xL -Z -xe -m0 -n \$*
            \end{enumerate}
\end{enumerate}

\section{关键字}

\begin{enumerate}
\item const
\item static
\item restrict
\item inline
\end{enumerate}

\section{简化编译}
使用Makefile,一个最小模板:

CFLAGS:= -O0 -g3 -I/usr/xx/include
LDFLAGS:=-L/usr/xx/lib
OUT:=test.out

all:\$(OUT)
$(OUT): src.c
    gcc -o $(OUT) $(CFLAGS) src.c $(LDFLAGS)

\section{编译选项}

\begin{enumerate}
\item -O 优化选项
\item -g 调试选项
\item -fPIC -fpic, 生成位置无关代码
\item -shared 生成动态库
\item -Wl,--no-undefined  禁止动态库编译时在找不到符号时生成动态库.
\item -Wl,-rpath=. -Wl,--enable-new-dtags
        指定程序运行时,依赖库的搜索路径,搜索路径具有优先级,可参考(man dlopen);
        SEARCH:LD\_LIBRARY\_PATH, RPATH, RUNPATH三者比较;
\item -Wl,-soname=libxx.so.1
        指定动态库的soname, 用于控制动态库版本.
\end{enumerate}

\section{工具链}

\begin{enumerate}
\item ldd
\item nm
        \begin{enumerate}
        \item  nm -C
                查看命名修饰(name mangling)前的名称.
        \item  nm |grep '\\<T>\\'
                查看实现的函数.
        \end{enumerate}
\item objdump
        \begin{enumerate}
        \item  objdump -DS
                反汇编
        \end{enumerate}
\item readelf
        \begin{enumerate}
        \item  readelf -d|grep SONAME
                查看动态库版本
        \item  readelf -d|grep RUNPATH
                 查看运行时库搜索路径
        \end{enumerate}
\item strip
        删除符号表

使用man手册浏览上述命令所有选项.
\end{enumerate}

\section{基本的BASH环境}
        \begin{enumerate}
        \item  export LD\_LIBRARY\_PATH=\$LD\_LIBRARY\_PATH:./
                添加运行时搜索路径
        \item ldconfig -d ./
                维护当前目录动态库版本.
        \item cp,scp, rsync
                拷贝文件,或通过网络拷贝文件,注意scp默认行为会破坏符号链接, 请使用rsync -av方式.
        \item ssh 通过网络登陆主机
                \begin{enumerate}
                \item 建议在\$HOME/.ssh/config 文件中配置主机IP, 用户名等;
                \item 使用ssh-copy-id上传密钥,节省密码输入操作;
                \end{enumerate}
        \item  chmod +x
                使用批处理脚本,增加可执行权限.
        \item ulimited -c unlimited
                产生core文件的开关
                或查看IPC的系统限制
        \end{enumerate}

\section{可配置的程序}
        \begin{enumerate}
        \item  命令行
                参考man 3 getopt
        \item 配置文件
                尽量采用标准格式,如xml, json, linux常用配置文件格式(libconfuse解析)的方式等.
        \item  环境变量
                man getenv
        \end{enumerate}

\section{libc}

\begin{enumerate}
\item 内存操作: malloc, free ,remalloc, memset, memcpy etc.
\item 字符串操作 strncpy, strcmp etc.
\item 文件操作 fopen, fread,
\end{enumerate}

参考:
\begin{enumerate}
\item 头文件位于/usr/include ,
\item 函数参数,返回值,特性,错误码请参考man手册.
\item https://www.gnu.org/software/libc/manual/html\_node/
\end{enumerate}

替代选择:
\begin{enumerate}
\item 内存管理: jemalloc, tcmalloc
\item 字符串操作, 注意越界
\item 文件操作建议使用系统调用open和mmap方式.
\end{enumerate}

\section{基础知识-pthread}

\begin{enumerate}
\item  线程操作, create, join, detach, exit
\item  锁,pthread\_mutex\_xxx, pthread\_spin\_xxx , pthread\_rwlock\_xxx
\end{enumerate}

\section{基础知识-IPC}

\begin{enumerate}
\item Sys V和Posix两种方式.
\item 信号量 sem\_xxx, semxxx
\item 共享内存 shm\_xxx, shmxxx
\item 消息队列  mq\_xxx msgxxx
\end{enumerate}
man pthreads,  man svipc
参考\it{Unix环境高级编程}

\end{document}
