% !TEX encoding = UTF-8
% !TEX program = lualatex

\documentclass{article}
\usepackage{color}
\usepackage{indentfirst}
\usepackage{luatexja-fontspec}
\setmainjfont{FandolSong}


\begin{document}

\title{CentOS升级Linux Kernel}
\author{xuxx}
\date{2016-08-25}
\maketitle

\section{下载源码}
\begin{enumerate}
\item 从https://www.kernel.org/上下载最新的longterm版本;
\item https://www.kernel.org/pub/linux/kernel/ 可下载历史的版本;
\end{enumerate}

\section{准备环境}
安装dracut,用于生成initramfs:
\begin{enumerate}
\item dracut-network
\item dracut-fips
\item dracut-config-generic
\item dracut-033-360
\item dracut-config-rescue-033-360
\end{enumerate}

安装ncures-devel用于menuconfig方式配置内核选项
\begin{enumerate}
\item ncurses-devel-5.9-13.20130511.el7.x86\_64
\item ncurses-5.9-13.20130511.el7.x86\_64
\item ncurses-base-5.9-13.20130511.el7.noarch
\item ncurses-libs-5.9-13.20130511.el7.x86\_64
\end{enumerate}

安装编译过程中可能需要的软件:
\begin{enumerate}
\item bc-1.06.95-13.el7.x86\_64
\item openssl-devel-1.0.1e-51.el7\_2.5.x86\_64
\end{enumerate}

\section{编译内核}
\begin{enumerate}
\item 进入内核目录, 使用make help查看支持的操作.
\begin{enumerate}
    \item 配置内核选项:
        \begin{enumerate}
        \item 可以将当前内核的配置, 比如/boot/config-3.10.0-327.el7.x86\_64拷贝到内核源码目录,修改名称为.config
        \item 执行make menuconfig 查看需要的内核模块开关已经打开.
        \end{enumerate}
    \item make all 编译bzImage,modules,vmlinux
    \item make firmware\_install 安装固件
          make modules\_install  安装驱动到/lib/modules/\{内核版本\}
    \item make install 安装内核到/boot目录
\end{enumerate}
\end{enumerate}

\section{更新启动选项}
\begin{enumerate}
\item 使用grub2-mkconfig,查看新的内核版本是否可以检测到;
\item 备份/boot/grub2/grub.cfg文件;
\item 使用grub2-mkconfig -o /boot/grub2/grub.cfg最终生成文件;
\end{enumerate}
重启后查看 /proc/version 查看内核版本是否已经更新.

\section{DEBUG}
当取启动出现问题时:
\begin{enumerate}
\item 修改/etc/default/grub 文件,将GRUB\_CMDLINE\_LINUX中的rhgb, quiet去掉,
        使内核启动过程中打印详细信息;
\item 执行执行上面一步,更新启动选项:grub2-mkconfig -o /boot/grub2/grub.cfg.
\end{enumerate}

\end{document}
